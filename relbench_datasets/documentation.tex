\documentclass{article}
\usepackage{graphicx} % Required for inserting images
\usepackage{url}
\title{RelBench Documentation}
\author{Alessandra Cicciarelli}
\date{December 2025}

\begin{document}

\maketitle

\section{Using RelBench to Download and Prepare Relational Datasets}
\label{sec:relbench-download}

RelBench provides a unified interface for working with
realistic relational datasets. 
Each dataset is implemented through a
\texttt{Dataset} class that includes a \texttt{make\_db()} method
responsible for downloading raw data (if needed), validating their
integrity, and constructing a relational \texttt{Database} object
consisting of multiple tables.

This section documents how to correctly use RelBench to download
datasets, describes issues that may arise during the download and
processing steps, and provides practical solutions.

\subsection{Basic Usage}
RelBench datasets are accessed through the
\texttt{get\_dataset()} utility:

\begin{verbatim}
from relbench.datasets import get_dataset

dataset = get_dataset("rel-amazon")   # or rel-stack, ...
db = dataset.make_db()
\end{verbatim}

The call to \texttt{make\_db()} performs three tasks:

\begin{enumerate}
    \item Downloads the raw data if not already present in the cache.
    \item Verifies file integrity (via SHA256 hashes when available).
    \item Constructs a \texttt{Database} with normalized relational tables.
\end{enumerate}

Once the database is built, its tables can be exported as CSV files:

\begin{verbatim}
for name, table in db.table_dict.items():
    table.df.to_csv(f"{name}.csv", index=False)
\end{verbatim}

\subsection{Download Mechanism and Caching}
RelBench relies on the \texttt{pooch} library to download and cache raw
files. The default download location can be inspected with:

\begin{verbatim}
import pooch
print(pooch.os_cache("pooch"))
\end{verbatim}

If a file is already cached but corrupted or mismatched, clearing the
cache is often necessary:

\begin{verbatim}
rm -rf <path returned by os_cache>/pooch/*
\end{verbatim}

\subsection{Common Issues and Solutions}

\subsubsection*{1. Missing Packages}
Some datasets require external libraries such as \texttt{tqdm} for
download progress bars or \texttt{pyarrow} for JSON parsing. A common
failure is:

\begin{verbatim}
ValueError: Missing package 'tqdm' required for progress bars.
\end{verbatim}

\noindent
\textbf{Solution:}

\begin{verbatim}
pip install tqdm pyarrow pooch
\end{verbatim}

\subsubsection*{2. Broken or Outdated Dataset URLs}
Some RelBench datasets (notably \texttt{rel-amazon}) originally relied
on URLs hosted at:

\url{https://datarepo.eng.ucsd.edu/mcauley_group/data/amazon_v2/}

This domain is no longer active, which results in errors such as:

\begin{verbatim}
HTTPError: 404 Client Error: Not Found for URL
\end{verbatim}

To fix this, the dataset wrapper must be updated to use the current
location of the Amazon Review Data:


\url{https://mcauleylab.ucsd.edu/public_datasets/data/amazon_v2/}


\subsubsection*{3. Incorrect File Paths}
Older versions of RelBench used directory names such as
\texttt{reviewSets/}, which no longer exist in the updated dataset
repository. The correct updated folders are:

\begin{itemize}
    \item \texttt{metaFiles2/} for metadata
    \item \texttt{categoryFilesSmall/} for 5-core review data
    \item \texttt{categoryFiles/} for full review data
\end{itemize}

Ensuring that the dataset loader points to these locations resolves
404-Not-Found errors during retrieval.

\subsubsection*{4. SHA256 Integrity Check Failures}
RelBench optionally validates file integrity using known SHA256 hashes:

\begin{verbatim}
ValueError: SHA256 hash of downloaded file [...] 
does not match the known hash
\end{verbatim}

This occurs when the upstream dataset updates its files, which is the
case for several Amazon Review subsets. The downloaded file is deleted
for safety.

\textbf{Solution:}
Manually compute the new hash:

\begin{verbatim}
sha256sum meta_Books.json.gz
\end{verbatim}

Replace the outdated hash in the RelBench loader, e.g.:

\begin{verbatim}
known_hashes = {
    "meta_Books.json.gz": "<new sha256>",
    "Books_5.json.gz": "<new sha256>"
}
\end{verbatim}

Alternatively, set \texttt{known\_hashes = \{\}} to disable integrity
checks (not recommended for long-term reproducibility).

\subsubsection*{5. Cache Inconsistencies}
If the local cache contains a partial or corrupted file, Pooch may
attempt to reuse it, causing unexpected failures or parse errors.

\textbf{Solution:}
Clear the cache and retry:

\begin{verbatim}
rm -rf ~/.cache/pooch/*
\end{verbatim}

\subsection{Validated Working Configuration for rel-amazon}

A working configuration for \texttt{rel-amazon} in 2025 requires the
following modifications:

\begin{enumerate}
    \item Update the URL prefix to:

    \url{https://mcauleylab.ucsd.edu/public\_datasets/data/amazon\_v2}

    \item Replace deprecated paths:
    \begin{itemize}
        \item \texttt{reviewSets/} $\rightarrow$ 
              \texttt{categoryFiles/}
        \item \texttt{metaFiles/} $\rightarrow$ \texttt{metaFiles2/}
    \end{itemize}
    \item Update SHA256 hashes when upstream files change.
\end{enumerate}

These modifications restore full functionality to the RelBench wrapper.

\end{document}
